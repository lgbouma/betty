	\enddata
	%
	\tablecomments{
		ESS refers to the number of effective samples.
		$\hat{R}$ is the Gelman-Rubin convergence diagnostic.
		Logarithms in this table are base-$e$.
		$\mathcal{U}$ denotes a uniform distribution,
		and $\mathcal{N}$ a normal distribution.
    Many of the $T_{13}$ statistics may be \texttt{nan} in the event of a
    grazing transit.
		 (1) The ephemeris is in units of BJDTDB.
		%  (2) Although $\mathcal{U}(0,1+R_{\rm p}/R_\star)$ is formally
		%  correct, for this model we assumed a non-grazing transit to enable
		%  sampling in $\log \delta$.
		%  (3) The eccentricity vectors are sampled in the $(e\cos\omega,
		%  e\sin\omega)$ plane.
		%  (4) The true planet size is a factor of $((F_1+F_2)/F_1)^{1/2}$
		%  larger than that from the fit because of dilution from Kepler
		%  1627B, where $F_1$ is the flux from the primary, and $F_2$ is that
		%  from the secondary; the mean and standard deviation of $R_{\rm
		%  p}=3.817\pm0.158\,R_{\oplus}$ quoted in the text includes this correction,
		%  assuming $(F_1+F_2)/F_1\approx 1.015$.
	}
	\vspace{-0.3cm}
\end{deluxetable*}
% hidden entry needed to get pdflatex to work
\ 
\end{document}
